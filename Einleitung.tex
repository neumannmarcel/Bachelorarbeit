\section{Einleitung}
Dieses Kapitel beleuchtet zuerst die allgemeine Problemstellung mit der sich in dieser Arbeit auseinander gesetzt wird. Anschließend wird die Anwendung ViRGOS erklärt, die als Grundlage für diese Arbeit dient. Abschließend wird die Struktur der kompletten Arbeit in Form der einzelnen Kapitel erklärt.

\subsection{Problemstellung und Motivation} \label{Problemstellung}
Aus eigener Erfahrung ist bekannt, dass sich Dinge leichter erlernen lassen wenn man sie anwendet, statt über sie zu lesen oder sie erklärt zu bekommen. Besonders praktische Prozeduren müssen durchgeführt und geübt werden, um vollkommen erlernt zu werden. Virtual Reality bietet hierbei vielversprechende Möglichkeiten um sowohl theoretisches Wissen durch Visualisierung, als auch praktisches Wissen durch Anwendung zu vermitteln. Mehrere Studien bestätigen das Potential, das Virtual Reality als Bildungswerkzeug birgt. ~\parencite{8864531} ~\parencite{8797755} Des Weiteren ermöglicht Virtual Reality die Kollaboration über größere Distanzen. In einer zunehmend globalisierten und digitalisierten Gesellschaft ist Zusammenarbeit mit Personen in anderen Standorten keine Seltenheit. Dies kann jedoch zu Problemen führen, wenn Unterstützung eines Experten mit spezifischem Fachwissen benötigt wird. Insbesondere bei praktischen Prozessen, die Demonstrationen bedürfen, stellt die Entfernung eine große Hürde dar. Bei diesen Unterrichtungs- und Fernwartungsszenarien über große Distanzen bietet Virtual Reality eine vielversprechende Abhilfe. ~\parencite{10.1145/2807442.2807497} Bei diesen asymmetrischen Kollaborationen verfügt einer der Benutzer über größeres Wissen als ein oder mehrere andere. Dieser Benutzer fungiert als Lehrer und versucht sein Wissen mittels der Anwendung an die anderen Benutzer weiterzugeben. \\
 

\subsection{Ziel der Arbeit}

Das Ziel der Arbeit ist die bestehende Anwendung der Hochschule zu erweitern und mehreren Benutzern zeitgleich Zugriff zu gewähren. Die Benutzer sollen hierbei miteinander Interagieren und asymmetrisch Kooperieren können, um eine gemeinsame Aufgabe zu lösen. Die Asymmetrie soll durch die zwei Rollen Kommandant und Astronaut hervorgerufen werden, die die Benutzer einnehmen sollen. Der Kommandant soll den Astronauten als Experte bei der Aufgabe unterstützen. Hierfür sollen dem Kommandanten mehrere Werkzeuge zur Kommunikation zur Verfügung stehen. \\

Hieraus ergibt sich folgende Forschungsfrage:

\begin{itemize}
\item[FF1] \label{FF1} Welche Funktionalitäten muss eine VR Anwendung bieten, um die asymmetrische Kollaboration zu ermöglichen?
\end{itemize}

Es soll also festgestellt werden, welche Funktionalitäten unabdingbar für die Kooperation sind. Außerdem soll untersucht werden, welche Werkzeuge den Benutzern am meisten helfen ihre Aufgabe gemeinsam zu lösen.


\subsection{Struktur der Arbeit}
Das erste Kapitel dient als Einleitung und schildert die Problemstellung, mit der sich in dieser Arbeit befasst wird. Außerdem wird ViRGOS als Grundlage für die Arbeit erläutert und der Ist-Zustand der Anwendung vor Beginn der Entwicklung erklärt. \newline

Der Stand der Wissenschaft bezüglich der Thematik wird in Kapitel zwei beschrieben. Hier wird auf verwandte Literatur eingegangen, die sich mit dem Einsatz von Virtual Reality zur Vermittlung von Wissen und der Kooperation mehrerer Nutzer beschäftigt. \newline

Kapitel drei beschäftigt sich mit der Konzeption der Anwendung. Zu Beginn werden die Anforderungen an das Projekt analysiert. Anschließend wird eine Auswahl aus mehreren Technologien getroffen, die in diesem Projekt zum Einsatz kommen könnten. Im Anschluss wird das Konzept eines Prototypen und der eigentlichen Anwendung erklärt. Es wird auf die Mehrspielerfähigkeit von ViRGOS eingegangen und geplante Werkzeuge beschrieben. Zum Abschluss wird auf Tests der Anwendung und Befragungen von Probanden eingegangen. \newline

Wie die geplanten Funktionalitäten umgesetzt wurden ist Thema des vierten Kapitels. Als Erstes wird erklärt wie der Prototyp entwickelt wurde und zu was dieser fähig ist. Im nächsten Unterkapitel werden die Implementierungen der Funktionen von ViRGOS geschildert. Abschließend wird auf Einschränkungen eingegangen, die durch die Gesundheitslage zum Zeitpunkt der Erstellung dieser Arbeit bestand. \newline

Kapitel 5: Probandenbefragung \todo{Erklären} \newline

Abschließend wird im sechsten Kapitel das Fazit der Arbeit gezogen. Die Diskussion beinhaltet Diskrepanzen zwischen der anfänglichen Planung und der tatsächlichen Durchführung. Des Weiteren wird ein Ausblick auf mögliche Erweiterungen gegeben.
