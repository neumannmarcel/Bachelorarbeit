\section{Evaluation} \label{Evaluation}
Im Rahmen dieser Thesis wurde eine Probandenbefragung durchgeführt um Rückmeldung bezüglich der Soft- und Hardware zu bekommen. Ursprünglich sollte, wie in Kapitel \ref{Covid} beschrieben, eine größere Befragung durchgeführt werden, welche Aufgrund der derzeitigen Umstände nicht möglich war. Somit musste der Ablauf des Versuchs abgeändert werden, wodurch manche Fragen nicht optimal auf den Test zugeschnitten waren oder komplett obsolet wurden. 

\subsection{Versuchsablauf}
Die Befragung wurde in einem kleineren Kreis mit vier Probanden durchgeführt. Da das Ziel der Anwendung die Kooperation von zwei Benutzern ist, wurden die Probanden in zwei Teams eingeteilt. Ein Benutzer nahm die Rolle des Astronauten ein und betrat den Virtualizer. Außerdem trug der Proband das Gurtzeug, welches mit dem Kransystem verbunden wurde. Zuletzt wurde der Proband mit dem Virtual Reality Headset und den dazugehörigen Controllern ausgestattet. Aufgrund von technischen Komplikationen konnte der Voicechat leider nicht verwendet werden, weshalb normal im Raum kommuniziert wurde.\\

Währenddessen wurde dem Kommandanten jedes Teams die Aufgabe erklärt die der Astronaut lösen soll. Weder die Astronauten noch die Kommandanten kamen vor dem Test in Kontakt mit der Anwendung. Das bedeutet, dass sich die Astronauten komplett auf die Leitung durch den Kommandanten verlassen mussten. Gleichzeitig mussten die Kommandanten die Erklärung die sie erhalten haben in den dreidimensionalen Raum transferieren, um sich zurecht zu finden. Während der Durchführung des Tests mit dem ersten Team, wartete der Astronaut des zweiten Teams außerhalb des Raums um die Lösung der Aufgabe nicht im Voraus zu sehen.\\

Zu Beginn des Tests leiteten die Kommandanten die Astronauten durch den Kontrollraum der Rakete. Diese mussten an der Schalttafel den Druck im Raum ablassen, dann die Türe öffnen und den Aufzug betreten. Über den Aufzug verließen die Astronauten die Rakete. Bisher hatten die Kommandanten direkten Blickkontakt zu den Astronauten, da sie sich im selben virtuellen Raum befanden. Sobald der Astronaut die Rakete verlassen hat, griffen die Kommandanten auf den Bildschirm im Kontrollraum und das Duplizieren der Kamera zurück. Sobald die Astronauten auf der Oberfläche des Mondes angekommen waren, wurden diese von den Kommandanten durch den Versuchsaufbau geleitet. Sobald die Aufgabe abschlossen wurde, war der Test beendet und die Probanden wurden ausgetauscht. Während das zweite Team den Versuch durchführte, füllte das erste Team jeweils einen der Fragebögen aus, die sich im Kapitel Anhang befinden. 

\subsection{Testergebnisse}
Von den vier Probanden die befragt wurden, waren zwei männlich und zwei weiblich. Das durchschnittliche Alter der Testpersonen lag bei 25,5 Jahren. Nur einer der Probanden hatte Erfahrung mit Virtual Reality, während die restlichen zumindest wussten worum es sich handelt.\\

Die Fragen des Fragebogens sind nach den Bereichen \glqq ViRGOS\grqq{} und \glqq Kooperation\grqq{} sortiert. Um die Übersichtlichkeit bei der Auswertung zu wahren, werden zuerst die Fragen betrachtet die nur von den Astronauten beantwortet werden konnten. Anschließend werden die Fragen an die Kommandanten, und schließlich die Fragen an alle Teilnehmer ausgewertet. 

Die erste Frage die sich mit der Basissoftware ViRGOS beschäftigt konnte nur von den zwei Probanden beantwortet werden, die in der Rolle des Astronauten waren. Die Frage beschäftigt sich mit dem Grad der Immersion von ViRGOS. Die Astronauten bewerteten diesen mit \glqq Mittel\grqq{} und \glqq Eher hoch\grqq{}. Somit scheint es eine solide Grundlage mit Raum für Verbesserungen zu geben. Hierbei ist zu beachten, dass der Voicechat aus technischen Gründen nicht verwendet werden konnte. Dieser könnte sich auf den Grad der Immersion auswirken, weshalb hier weitere Tests notwendig sind.\\

Die Bedienbarkeit des Virtualizers wurde sowohl mit \glqq Eher hoch\grqq{}, als auch mit \glqq Eher niedrig\grqq{} beschrieben. Der Proband der mit der Bedienbarkeit unzufriedener war, hat angemerkt dass empfindlichere Sensoren notwendig wären um die Bewegung in VR zu verbessern. Bei dieser Diskrepanz und der kleinen Anzahl an Probanden sind Fehler in der Bedienung von Seiten der Anwender allerdings nicht auszuschließen.\\

\newpage

Die Unterstützung der Immersion durch die Gravitationssimulation wurde mit \glqq Mittel\grqq{} und \glqq Eher niedrig\grqq{} bewertet. Somit scheint es hier größeren Verbesserungsbedarf zu geben.\\

Die erste Frage an die Kommandanten beschäftigte sich mit den Werkzeugen die für die Kooperation zur Verfügung standen. Die Probanden sollten die Kombination von Werkzeugen angeben die für die Kooperation am hilfreichsten war. Durch die geringe Anzahl an Probanden konnten leider keine unterschiedlichen Kombinationen getestet werden, weshalb beide Kommandanten alle Werkzeuge zur Verfügung hatten. Zudem waren nicht alle Werkzeuge implementiert die in der Frage zur Auswahl standen. Die Antworten der Kommandanten waren zwei mal \glqq Bildschirm im Kontrollraum\grqq{} und ein Mal \glqq Übernahme der Kamera\grqq{}. Aufgrund der vielen Einschränkungen lassen sich leider nicht viele Schlüsse aus diesen Antworten ziehen. Es lässt sich aber sagen dass es für die Kommandanten notwendig ist, zu sehen wo sich der Astronaut befindet um sich in der Welt zu orientieren. Bei dem Test der Funktionen kam leider auch ein Bug zum Vorschein, der aufgrund der fehlenden Hardware zuvor nicht aufgefallen ist. Wenn ein Kommandant die Kamera eines Astronauten übernimmt, folgt diese zwar der Position aber nicht der Rotation des anderen Benutzers. Somit ist es sehr schwer zu erkennen wo sich der Astronaut gerade befindet. Dieser Fehler wurde von den Probanden bemängelt und sollte in Zukunft behoben werden.\\

Die Frage die an alle Probanden gestellt wurde, war eine Einschätzung der Bedienbarkeit der Software insgesamt. Hier waren die Antworten drei mal \glqq Mittel\grqq{} und ein mal \glqq Eher hoch\grqq{}. Hierbei ist zu beachten dass die Astronauten noch nie einen Virtualizer bedient haben. Zudem hatten die Kommandanten nach eigener Aussage keinerlei Erfahrungen mit der Steuerung eines Avatars zum Beispiel im Rahmen eines Videospiels. Vorgeschlagene Verbesserungen für die Bedienbarkeit waren größere Beschriftungen der Knöpfe in der Kommandozentrale. 

\newpage

Dieses Problem kann eventuell gelöst werden, indem ein größerer Bildschirm für die Kommandanten verwendet wird. In diesem Test stand den Kommandanten ein Laptop mit relativ kleiner Bildschirmdiagonale zur Verfügung, was Details schwerer erkennbar machte. Dies könnte auch eine Erklärung für die persönliche Beobachtung sein, dass beide Kommandanten sofort die Kamera übernommen haben statt dem Astronauten auf dem Bildschirm zu folgen, da dieser nochmals kleiner ist.\\

Weitere Anmerkungen der Probanden waren, dass sich die Kabel die das Headset mit dem Computer verbinden, störend auf die Benutzung auswirken. Dieses Problem ist nur durch eine Änderung der Hardware lösbar, zum Beispiel durch den Einsatz eines kabellosen VR Headsets. Ein weiterer Vorschlag war die Darstellung von Arbeitsanweisungen in der Anwendung selbst, da sich die Kommandanten den Ablauf nicht komplett merken konnten. Wenn die Benutzer der Anwendung beide Laien sind, ist dies sinnvoll um zusätzliche Eingriffe und Erklärungen von außen zu vermeiden. Dies ist allerdings nicht notwendig, falls die Anwendung für eine Art der Fernwartung eingesetzt wird, bei der der Kommandant ein Experte ist.\\

Eine weitere persönliche Beobachtung war, dass Benutzer die das VR Headset verwendeten, sehr ungestüm mit ihren Interaktionen waren. Manche Knöpfe in der virtuellen Welt ändern ihre Farbe nach einer bestimmten Zeit oder spezifischen Aktionen um dem Benutzer zu signalisieren, dass sie nun aktiviert werden können. Beiden Astronauten schienen diese Änderungen nicht aufzufallen, wodurch es teilweise zu Verwirrung kam wenn gedrückte Knöpfe nichts auslösten. Eventuell müssen hier offensichtlichere Zeichen gegeben werden, dass die Anwendung arbeitet und der Benutzer warten muss. Alternativ kann dies aber auch auf den Charakterzügen der Astronauten basieren, was aufgrund der begrenzten Probandenzahl nicht erwiesen werden kann.