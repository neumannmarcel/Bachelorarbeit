\section{Fazit}
In diesem Kapitel wird im Rahmen der Diskussion zuerst auf die Anforderungen eingegangen. Hierbei wird beschrieben ob die Anforderung komplett, nur teilweise oder gar nicht umgesetzt werden konnte. Zum Abschluss werden im Unterkapitel Ausblick mögliche Erweiterungen und aktuelle Fehler in der Anwendung genannt die in Zukunft implementiert und verbessert werden könnten.

\subsection{Diskussion} \label{Diskussion}

Die meisten Anforderungen die in der Anforderungsanalyse definiert wurden konnten umgesetzt werden. Die Anforderung A1 \glqq Mehrere Benutzer gleichzeitig in der Anwendung\grqq{} wurde komplett umgesetzt. Die Anwendung kann von einem bis zehn Spielern gleichzeitig verwendet werden. Die Anwendung reagiert dynamisch auf das Beitreten und das Verlassen von Benutzern, ohne dass etwas angepasst oder neu gestartet werden muss.\\
Anforderung A2 \glqq Keine technische Abhängigkeit von einem VR-Headset\grqq{} wurde ebenfalls erfolgreich umgesetzt. Die Anwendung kann mit einem VR-Headset und dem Virtualizer oder über Maus und Tastatur benutzt werden. Welcher Spieler welche Technologie benutzt ist hierbei auch frei kombinierbar.\\
Die Rollen die in A3 \glqq Astronaut und Kommandant als Rollen für die Benutzer\grqq{} definiert wurden, wurden auch umgesetzt. Den Spielern werden keine festen Rollen zugewiesen, sondern von den Spielern dynamisch selbst gewählt. Dies geschieht durch die Positionierung in der Anwendung und die Benutzung der Werkzeuge, die für den Kommandanten gedacht sind.\\
A4 \glqq Informationsübertragung zwischen den Benutzern\grqq{} hingegen nicht komplett umgesetzt werden. Die sprachliche Kommunikation wurde bereits von der Basisanwendung angeboten. Die Position von anderen Spielern kann über den Bildschirm im Kontrollraum ermittelt werden. Mit der Funktion, die die Übernahme der Sicht eines anderen Spielers ermöglicht, können viele visuelle Informationen an den Kommandanten übermittelt werden. Die Funktionen für den Verweis auf Punkte im Raum und das Anbringen von Notizen konnten jedoch nicht umgesetzt werden. Abgesehen von dem Voicechat findet die Kommunikation aktuell folglich nur in eine Richtung von dem Astronauten zu dem Kommandanten statt.\todo{Forschungs- frage beantwortet? Erkenntnisse Umfrage?}

\subsection{Ausblick} \label{Ausblick}

Die Anwendung kann auf mehrere Arten erweitert und verbessert werden. Der Versuchsaufbau kann komplexer gestaltet werden, damit mehr Kooperation und Anleitung von Seiten des Kommandanten notwendig ist. Die Knöpfe die aktuell zur Anwendung kommen sind recht selbsterklärend, weshalb andere Eingabemöglichkeiten wie zum Beispiel Hebel, Schalter oder Tastenfelder verwendet werden könnten. Des Weiteren kann eine andere Art des Eingreifens für den Kommandanten implementiert werden. Aktuell kann der Kommandant über den Bildschirm die Sicht des Astronauten übernehmen, was ausreichend ist um den Astronauten durch die Aufgabe zu navigieren. Eine Erweiterung, die den Kommandanten zwingen würde diese Ansicht zu verlassen, wäre hilfreich um die Interaktivität für den Kommandanten zu steigern. Es gibt mehr Vorteile dass sich der Kommandant auch in der Anwendung befindet, wenn dieser den Astronauten durch aktives Eingreifen unterstützen muss.\\

Auch auf technischer Seite können Verbesserungen vorgenommen werden. Die Kollisionsvermeidung, die aktuell mit der Stoppuhr gelöst wird, führt zu manchen Problemen. Wenn mehrere Spieler gleichzeitig Knöpfe betätigen möchten wird dies von den zentralen Skripten als Kollision interpretiert und verhindert. Auch die Verwendung von Photons PUN 2 bietet Raum zur Verbesserung. Zwar bietet PUN alles was für die erfolgreiche Umsetzung der Anwendung notwendig ist, jedoch wird hierbei auf den Cloud-Service von Photon zurück gegriffen. Somit bestehen gewisse Abhängigkeiten und potentielle rechtliche Fragen, die umgangen werden könnten. Eine Alternative wäre Bolt, was ebenfalls von Photon entwickelt wurde. Statt auf einen Cloud-Service zuzugreifen würde hierbei einer der Clients als Host eingesetzt werden. Diesem Host würden die anderen Clients dann beitreten. Diese Variante würde unweigerlich neue Probleme aufwerfen, aber eine Unabhängigkeit von den Cloud-Servern von Photon bieten.\\

Wie in Kapitel \ref{Konzeption} erwähnt waren Werkzeuge für den Kommandanten geplant die leider nicht umgesetzt werden konnten. Somit kann die Anwendung um ein Werkzeug erweitert werden, mit dem die Benutzer auf Punkte im Raum verweisen können. Auch das Platzieren von Notizen kann hilfreich sein, wenn die Aufgabe hiervon Gebrauch machen kann. Auch über die in dieser Arbeit konzeptionierten Werkzeuge hinaus gibt es Möglichkeiten der Erweiterung. Die Knöpfe, die aktuell verwendet werden, fallen durch ihre leuchtenden Farben schnell auf, was die Interaktivität nahe legt. Falls weniger auffällige Objekte verwendet werden, wäre ein Werkzeug hilfreich, mit dem interaktive Objekte hervorgehoben werden. Ein Ansatz hierfür wurde in Kapitel \ref{Vorstudie} erwähnt. Zuletzt gibt es einen Bug in dem bestehenden Werkzeug mit dem die Kamera übernommen werden kann. Der in Kapitel ~\nameref{Evaluation} beschriebene Fehler verhindert eine Rotation der Kamera, wodurch die Orientierung in der Welt sehr viel schwerer fällt und die Nützlichkeit der Werkzeugs extrem eingeschränkt wird.