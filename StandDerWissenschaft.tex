\section{Stand der Wissenschaft} \label{Wissenschaft}
Dieses Kapitel präsentiert den Stand der Wissenschaft im Bezug auf kooperatives Lernen in Virtual Reality. Zunächst wird auf das Lernen in Virtual Reality allgemein eingegangen und welche Vorteile es zu klassischem Lernen bietet. Anschließend werden Arbeiten behandelt, die sich ebenfalls mit Kooperation in virtuellen Räumen beschäftigen.

\subsection{Lernen in Virtual Reality}
In diesem Kapitel wird sich mit den allgemeinen Möglichkeiten und Vorteilen auseinandergesetzt, die Virtual Reality im Bezug auf den Lernprozess haben kann. Zuerst werden positive Effekte aufgezeigt, die Virtual Reality auf den Lernprozess haben kann. Anschließend wird auf die Theorie der Lerntypen eingegangen, welche dann mit virtuellen Umgebung in Verbindung gebracht werden. Zum Schluss wird eine Theorie der Unterteilung von virtuellen Welten in sogenannte Lernwelten behandelt, welche die Herangehensweise an den Lernprozess, abhängig von dem Ziel der Anwendung, beschreiben.

\subsubsection{Positive Lerneffekte von Virtual Reality}
Virtual Reality bietet viele Vorteile gegenüber klassischen Lernansätzen. Besonders bei praxisorientierten Aufgaben bieten Virtual Reality Anwendungen die Möglichkeit, Bewegungsabläufe nachzuahmen und praktisch anzuwenden. Wenn sich der Lernende rein theoretisches Wissen aneignet ist zusätzlich eine Transferleistung notwendig um dieses Wissen in die Praxis umzusetzen. Zudem ermöglichen diese Anwendungen eine Simulation des Kontextes, in dem das Wissen und die Fähigkeiten des Lernenden gebraucht werden. So kann die spätere Arbeitsumgebung eines Benutzers simuliert werden, wodurch erneut keine Transferleistung von der Trainings- in die Arbeitsumgebung erbracht werden muss.\\
\newpage

Zusätzlich zu praktischem, kann auch theoretisches Wissen in VR vermittelt werden. Durch das Verknüpfen von theoretischen Informationen mit dem Erlebnis in der 3D-Welt steigert sich das Erinnerungsvermögen der Lernenden. Das räumliche Vorstellungsvermögen verbessert sich durch den Kontakt mit den dreidimensionalen Darstellungen ebenfalls ~\parencite{Buehler2020}.\\

Auch auf wirtschaftlicher Seite bietet VR großes Potential. Wenn größere Gruppen geschult oder unterrichtet werden müssen gibt es das Problem, dass es oft nur begrenztes Trainingspersonal oder Übungsmaterialien gibt. Besonders die Schulung für den Umgang mit größeren Geräten kann problematisch sein, da es zu kostspielig ist ein Übungsgerät für jeden Lernenden bereitzustellen. Wenn sich die Lernenden allerdings mit demselben Gerät abwechseln, erhöht sich die Dauer der Schulung immens. Die Verwendung von VR bietet die Möglichkeit die Lernenden unabhängig von der Gruppe zu schulen. Jeder Benutzer kann hierbei in seinem eigenen Tempo lernen, ohne den Rest der Gruppe auszubremsen oder abzuhängen. Da die Anwendungen so entworfen werden können, dass der Benutzer selbstständig zurecht kommt, muss das Schulungspersonal nur in Ausnahmefällen einschreiten. Somit können Kosten gespart werden, da weniger Personal für größere Gruppen eingesetzt werden muss.\\

Weitere Kosten können gespart werden, da keine Übungsmaschinen angeschafft, oder aus dem laufenden Betrieb entfernt werden müssen. Zudem besteht nicht die Gefahr, dass Benutzer das Übungsgerät falsch verwenden und somit beschädigen. Die Lernenden sind darüber hinaus unabhängig von den Örtlichkeiten. Schulungen können somit überall durchgeführt werden, sofern die notwendige Hardware vorhanden ist. Auch eine Verletzungsgefahr ist so gut wie ausgeschlossen. Somit können Tätigkeiten, die in der Regel gefährlich oder beschwerlich sind, gefahrlos geübt werden ~\parencite{Buehler2020}.

\subsubsection{Lerntypen}
Jeder Lernende hat unterschiedliche Herangehensweisen an den Prozess des Lernens und verschiedene Vorlieben. Trotz der zentralen Wichtigkeit des Lernens im Bezug auf schulische und akademische Leistungen, aber auch darüber hinaus, gibt es viele, teils widersprüchliche Arbeiten, die sich mit der Thematik beschäftigen. Ein weit verbreiteter Ansatz ist das sogenannte VARK-Modell. Hierbei werden vier Gruppen von Lernenden, anhand ihrer bevorzugten Art der Präsentation von Daten, unterschieden:
\begin{itemize}
\item Visuell: Der Lernende bevorzugt es Daten in Form von Bildern, Diagramme usw. zu sehen 
\item Auditiv: Dem Lernenden liegt das Hören oder Aussprechen bzw. Debattieren über Daten am Meisten
\item Lesen/Schreiben: Der Lernende versteht Daten am Besten indem er/sie Texte oder Tabellen liest und Notizen anfertigt
\item Kinästhetisch: Der Lernende wendet gelernte Daten am Liebsten praktisch an um diese zu verinnerlichen
\end{itemize}
Laut dem VARK-Modell können Lernende nicht nur eine Vorliebe für eine dieser Methoden, sondern für mehrere gleichzeitig aufweisen. \\

In einer Studie ~\parencite{8971204} die sich mit der Thematik beschäftigt, wurde festgestellt, dass der Großteil der Lernenden eine Mischung aus allen vier Methoden bevorzugt. Außerdem ist zu dem Schluss gekommen worden, dass Lernende selbst dann keine besseren Lernerfolge vorweisen können, wenn sie exklusiv in ihrer dominanten Lernmethode unterrichtet wurden. Dies verwischt die Grenzen der vier Gruppen und legt eine multimodale Art des Lernens nahe.
Dies wird weiterhin durch Befunde gestützt, die erschließen lassen, dass die Art der Reizverarbeitung ausschlaggebend ist und nicht die Modalität der Reizaufnahme ~\parencite{UBHD-67741817}. Beispielsweise wird Sprache ähnlich verarbeitet wenn sie auditiv oder in geschriebener Form aufgenommen wird. Bilder und Texte hingegen werden visuell wahrgenommen, aber intern unterschiedlich verarbeitet.\\

Basierend auf der Vielzahl von unterschiedlichen Ansätzen ist zu sagen, dass sich sowohl die Präferenzen des Lernen, als auch die Bedingungen der Lernumgebung ausschlaggebend auf den Lernerfolg auswirken ~\parencite{UBHD-67741817}. Virtual Reality stellt durch die potentielle Abdeckung aller Lernmethoden und Sinneskanäle eine vielversprechende Alternative zu klassischem Lernen dar. 

\subsubsection{Interaktion und Lernwelttypen}
Der wohl größte Vorteil, den Virtual Reality im Bezug auf Lernen hat, ist die Möglichkeit der Interaktion mit der virtuellen Welt. Es wurde bereits bestätigt, dass Interaktivität zu einem höheren Engagement bei der Auseinandersetzung mit den Inhalten führt ~\parencite{8864531}. Besonders bei prozeduralem Wissen, also dem Erlernen von Handlungsverläufen, stellte sich die Interaktivität in Virtual Reality als effektives Lehrmittel heraus ~\parencite{8797755}. \newline

Die Art und der Umfang der Interaktivität ist jedoch stark von der Thematik und dem Ziel der Virtual Reality Anwendung abhängig. So können sich virtuelle Lernwelten in 4 verschiedene Gruppen einteilen lassen ~\parencite{UBHD-65563883}:

\begin{itemize}
\item Explorationswelten
\item Trainingswelten
\item Experimentalwelten
\item Konstruktionswelten
\end{itemize}

Ziel der Explorationswelten ist die freie, möglichst uneingeschränkte Begehbarkeit der Welt. So kann beispielsweise eine antike Stadt virtuell nachgebaut und für den Benutzer frei begehbar gemacht werden. Der Benutzer entscheidet hierbei selbst über die Geschwindigkeit und Reihenfolge in der die Elemente der Welt abgearbeitet werden. \\

Bei Trainingswelten steht besonders das Vermitteln von prozeduralen Fertigkeiten im Mittelpunkt. So können Trainingswelten für den Umgang mit Maschinen entwickelt werden. Der Benutzer ist in der Lage sich mit dem Gerät auseinanderzusetzen und die notwendigen Handgriffe zu üben, ohne dass die potentiell teure Maschine tatsächlich angeschafft werden muss. Auch in Situationen die den Lernenden in Gefahr bringen könnten ist vorherige Übung in einer Trainingswelt wie einem Fahrsimulator von Vorteil. \\

Experimentalwelten vermitteln im Gegensatz zu Explorationswelten keine statischen Fakten, sondern versuchen dem Lernenden durch Experimente die geltenden Gesetzmäßigkeiten der virtuellen Welt zu erläutern. Der Benutzer kann zum Beispiel Eigenschaften von Objekten oder der Welt verändern um die Auswirkungen dieser Änderungen zu simulieren und zu verstehen. Experimentalwelten dienen also auch als wissenschaftliches Werkzeug zur Bestätigung oder Wiederlegung eigener Hypothesen. \\

Konstruktionswelten gehen im Vergleich zu Experimentalwelten einen Schritt weiter und verlassen sich mehr auf die Kreativität des Lernenden. In Konstruktionswelten kann der Benutzer die virtuelle Welt selbst nach seinen Vorstellungen oder einer Aufgabenstellung gestalten. Ähnlich wie das bekannte Papierbrücken-Experiment, bei dem Schüler oder Studenten eine möglichst stabile Brücke aus Papier bauen sollen, wird dem Benutzer in einer Konstruktionswelt eine Aufgabe gestellt die umgesetzt werden soll. Mögliche Fehleinschätzungen fallen anhand der laufenden Simulation sofort auf und können angepasst werden. Somit wird sofort aus dem Erfolg oder Misserfolg der eigenen Konstruktion gelernt. Da der Benutzer die Konstruktion selbst durchgeführt hat, zählt hier nicht nur das Endergebnis, sondern auch der Weg, über den zu diesem gelangt wurde. \\

\subsection{Kooperation in Virtual Reality}
In diesem Unterkapitel wird speziell auf die Kooperation in Virtual Reality eingegangen. Zu Beginn wird die Interaktion zwischen mehreren Benutzern thematisiert. Anschließend wird auf die unterschiedlichen Arten der Kommunikation zwischen diesen Benutzern eingegangen. Zum Schluss wird der Sonderfall der asymmetrischen Kooperation sowie die unterschiedlichen Arten der Asymmetrie an mehreren Beispielen erklärt.

\subsubsection{Interaktion mit mehreren Benutzern}
Mit der konstanten Weiterentwicklung der Hardware, als auch der Software von Virtual Reality, erweitern sich auch die Funktionalitäten und Anwendungsbereiche der Technologie. Eine dieser Erweiterungen ist die Anwendung von VR mit mehreren kooperierenden Benutzern. Bereits als die Technologie in den 90er Jahren noch in den Kinderschuhen steckte wurden erste Tests durchgeführt, die das Potential von virtuellen Mehrspieler-Anwendungen beschreiben ~\parencite{youngblut1998educational}.\newline
In der Zwischenzeit fand VR für Kooperation mit mehreren Benutzern einige Anwendungen. Virtual Reality ermöglicht die realistische Darstellung von Objekten aus unterschiedlichen Blickwinkeln. Durch die Darstellung der anderen Benutzer und deren Interaktion mit der virtuellen Umwelt wird eine soziale Erfahrung erzeugt, die Zusammenarbeit und Diskussion über die Thematik ermöglicht ~\parencite{8798289}. Benutzer sind hierbei nicht auf lokale Zusammenarbeit beschränkt, sondern können über große Distanzen kooperieren. Diese Distanz stellt bei Interaktionen in Virtual Reality kein großes Problem dar und kann die Zusammenarbeit sogar erleichtern. Laut einer Studie ~\parencite{8848001} ist die Kooperation in Virtual Reality effektiver wenn die Teilnehmer physikalisch räumlich getrennt sind, statt sich im selben Raum aufzuhalten. 

\subsubsection{Kommunikation}
Ein wichtiger Punkt bei Kooperation in einer virtuellen Umgebung ist die Kommunikation mit anderen Benutzern. Die grundlegende Art der Kommunikation, die von den meisten Anwendungen angeboten wird, ist die gesprochene Sprache über Voice-Chats. Mit anderen Benutzern sprechen zu können ermöglicht bereits den Austausch von vielen Informationen und ist die intuitivste Art der Kommunikation. Gerade in dreidimensionalen Welten verfügt jeder Benutzer über seinen eigenen Blickwinkel auf dasselbe Objekt, wodurch es zu einer Diskrepanz der räumlichen Informationen kommen kann. Ein Benutzer kann verbal auf etwas hinweisen, das ein anderer aufgrund seiner räumlichen Position nicht sehen kann. Jedoch kann die verbale Beschreibung eines Punktes im Raum vage ausfallen, wodurch es den anderen Benutzern schwer fällt zu erkennen was gemeint ist. Eine Möglichkeit dies zu lösen ist die Verwendung von Werkzeugen, die non-verbale Kommunikation erlauben. Bei einer beispielhaften Anwendung\footnote{https://weare-rooms.com/vr-cad/ (Zugriff: 11.05.2020)}, die für die Zusammenarbeit an 3D-Modellen und Dokumenten entworfen wurde, erhält man die Möglichkeit mit seiner virtuellen Hand auf Punkte im Raum zu zeigen. Zusätzlich können Notizen geschrieben und an Objekten platziert werden, um Informationen mehrmals zugänglich zu machen. Außerdem kann auf Objekten gezeichnet werden, um so beispielsweise interessante Punkte zu markieren. Dies sind nur einige der Möglichkeiten, mit denen die Kommunikation mit anderen Benutzern erweitert werden kann.

\subsubsection{Asymmetrische Kooperation}
Eine besondere Unterart der Kooperation in Virtual Reality ist die asymmetrische Kooperation. Asymmetrie bedeutet in diesem Fall, dass es Unterschiede bei der Benutzung der Anwendung gibt. Diese Asymmetrie kann sich hierbei auf unterschiedliche Bereiche beziehen. Zum Beispiel kann es eine Asymmetrie in der Technologie geben, mit der die virtuelle Welt betreten wird. So kann ein Benutzer die Welt mit einem VR-Headset betreten, während ein anderer eine Augmented Reality Technologie oder einen einfachen Desktop-Computer verwendet. Diese Differenz kann Vorteile, aber auch Nachteile in der Kollaboration bringen und ist stark von der Art und dem Ziel der Anwendung abhängig ~\parencite{8798080}. \\

Eine Asymmetrie kann aber auch in Form von unterschiedlichen Perspektiven in derselben Anwendung auftreten. In einer Studie ~\parencite{7563562} wurde eine Anwendung vorgestellt, in der mehrere Benutzer in unterschiedlichen Rollen an einer gemeinsamen Aufgaben arbeiten können. Diese Rollen verfügen über verschiedene Skalierungen der virtuellen Welt, was andere Blickwinkel auf dieselben Objekte ermöglicht. Somit ist ein Benutzer ein Riese, der die komplette Szene von oben herab sieht, während der andere winzig ist und sich zwischen kleinen Objekten in der Szene bewegen kann. Aus beiden Blickwinkeln können Informationen zur Lösung der Aufgabe gesammelt werden, die sich gegenseitig ergänzen. \\

Eine weitere Art der Asymmetrie zeichnet sich durch einen Unterschied in den Funktionalitäten aus, die den Benutzern zur Verfügung stehen. Diese Art der Kooperation kommt besonders häufig bei Lehrer-Schüler-Szenarios zum Einsatz. Die lehrende Person verfügt also über Funktionalitäten oder Werkzeuge, die zum Anleiten des Schülers verwendet werden können. Der Schüler hingegen benötigt keinen Zugriff auf diese Werkzeuge. 

\newpage

Ein Beispiel hierfür ist ObserVAR ~\parencite{8943686}, eine Anwendung bei der ein Lehrer mehrere Schüler in Form von Avataren beobachten kann. Dem Lehrer wird hierbei visuell angezeigt auf welchen Punkt im Raum jeder Schüler schaut. Somit kann die Aufmerksamkeit der Schüler gezielt gelenkt werden. Eine weitere Einsatzmöglichkeit ist die Planung eines Raumes. Hier kann sich ein Kunde einen virtuellen Raum vor dem Kauf ansehen. Der Immobilienhändler verfügt nun über die Möglichkeit virtuelle Möbel nach den Vorstellungen des Kunden im Raum zu platzieren ~\parencite{7223433}. \\

All diese unterschiedlichen Arten der Asymmetrie können in verschiedenen Kombinationen auftreten, abhängig von der Art der Kooperation und dem Ziel der Anwendung. In diesem Beispiel der Fernwartung ~\parencite{10.1145/2807442.2807497} kommen sowohl unterschiedliche Technologien als auch Funktionalitäten zum Einsatz. In dieser Anwendung lehrt ein Experte einem anderen Benutzer den Umgang mit einem physikalischen Objekt in Echtzeit. Damit der Benutzer den Anweisungen direkt folgen kann, muss dieser sowohl die Anweisung als auch das physikalische Objekt sehen. Dies wird durch eine Augmented Reality Brille ermöglicht. Der Experte hingegen verwendet eine Virtual Reality Brille und verfügt über Möglichkeiten Markierungen zu setzen und wichtige Informationen an den anderen Benutzer zu übermitteln. 
